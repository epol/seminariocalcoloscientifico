\documentclass[a4paper,10pt]{article}

\usepackage{amsmath}
\usepackage{amssymb}
\usepackage{amsthm}
\usepackage{xfrac}
\usepackage[all]{xy}
\usepackage{graphicx}
%\usepackage{fullpage}
\usepackage{hyperref}
\usepackage[utf8x]{inputenc}
\usepackage[italian]{babel}

%\setlength{\parindent}{0in}

\newcounter{counter1}

\theoremstyle{plain}
\newtheorem{myteo}[counter1]{Teorema}
\newtheorem{mylem}[counter1]{Lemma}
\newtheorem{mypro}[counter1]{Proposizione}
\newtheorem{mycor}[counter1]{Corollario}
\newtheorem*{myteo*}{Teorema}
\newtheorem*{mylem*}{Lemma}
\newtheorem*{mypro*}{Proposizione}
\newtheorem*{mycor*}{Corollario}

\theoremstyle{definition}
\newtheorem{mydef}[counter1]{Definizione}
\newtheorem{myes}[counter1]{Esempio}
\newtheorem{myex}[counter1]{Esercizio}
\newtheorem*{mydef*}{Definizione}
\newtheorem*{myes*}{Esempio}
\newtheorem*{myex*}{Esercizio}

\theoremstyle{remark}
\newtheorem{mynot}[counter1]{Nota}
\newtheorem{myoss}[counter1]{Osservazione}
\newtheorem*{mynot*}{Nota}
\newtheorem*{myoss*}{Osservazione}


\newcommand{\obar}[1]{\overline{#1}}
\newcommand{\ubar}[1]{\underline{#1}}

\newcommand{\set}[1]{\left\{#1\right\}}
\newcommand{\pa}[1]{\left(#1\right)}
\newcommand{\ang}[1]{\left<#1\right>}
\newcommand{\bra}[1]{\left[#1\right]}
\newcommand{\abs}[1]{\left|#1\right|}
\newcommand{\norm}[1]{\left\|#1\right\|}

\newcommand{\pfrac}[2]{\pa{\frac{#1}{#2}}}
\newcommand{\bfrac}[2]{\bra{\frac{#1}{#2}}}
\newcommand{\psfrac}[2]{\pa{\sfrac{#1}{#2}}}
\newcommand{\bsfrac}[2]{\bra{\sfrac{#1}{#2}}}

\newcommand{\der}[2]{\frac{\partial #1}{\partial #2}}
\newcommand{\pder}[2]{\pfrac{\partial #1}{\partial #2}}
\newcommand{\sder}[2]{\sfrac{\partial #1}{\partial #2}}
\newcommand{\psder}[2]{\psfrac{\partial #1}{\partial #2}}

\newcommand{\intl}{\int \limits}

\DeclareMathOperator{\de}{d}
\DeclareMathOperator{\id}{Id}
\DeclareMathOperator{\len}{len}

\DeclareMathOperator{\gl}{GL}
\DeclareMathOperator{\aff}{Aff}
\DeclareMathOperator{\isom}{Isom}

\DeclareMathOperator{\im}{Im}




\title{Seminario Calcolo Scientifico - appunti1}
\author{Enrico Polesel}
\date{\today}

\begin{document}
\maketitle

\section{Contest}
\label{sec:contest}

\subsection{Fredholm integral equations}
\label{sec:fredholm}

Many physical problems can be modeled with the equation
\begin{equation}
  \label{eq:fredholm}
  g(t) = \intl _{\Omega} k(t,s) f(s) \de s
\end{equation}
where $f(s)\in L^2(\Omega)$ is the function that we want to know and
$g(t)\in L^2(\Omega)$ is the measure we can get, $k(t,s) \in
L^2(\Omega \times \Omega)$ is called \textit{kernel} and it is given
by the model.

So we have the linear funcion
\[ \begin{matrix}
  K:\; &L^2(\Omega) &\longrightarrow &L^2(\Omega)\\
  & f &\longrightarrow & \intl _\Omega k(\cdot , s) f(s) \de s
  \end{matrix}
\]

For the Riemann-Lebesgue lemma we have that $K$ smooths the higher
frequency components of $f$, in fact if we have $\Omega = \mathbb{R}$
then
\[ \lim _{z \to \infty} \intl _\mathbb{R} k(t,s) e^{-zs} \de s = 0 \]

Under certain hypothesis ($K$ compact) we can write the Singualr Value
Expansion (SVE) of $K$, so there exist $\set{v_i}_{i\in \mathbb{N}}$
orthonormal basis of $L^2(\Omega)$, $\set{v_i}_{i\in \mathbb{N}}$
orthonormal basis of $\im(K)$, $\set{\sigma _i}_{i\in \mathbb{N}}$
nonnegative set of singualar values such that
\begin{equation}
  \label{eq:SVErealtion}
  K(v_j) = \sigma _j u_j
\end{equation}
We also have $\sigma _i > \sigma _{i+1}$ and $i \to \infty \Rightarrow
\sigma _i \to 0$.

So we can write
\begin{equation}
  \label{eq:SVE}
  K(f) = \sum _{j\in \mathbb{N}} u_j \sigma _j \ang{f,v_j}
\end{equation}

It is known that with the increasing of $j$ the functions $v_j$ and
$u_j$ becomes more and more oscillating, the fact that $\sigma _j \to
0$ dumps that high frequency components when applicating $K$.


If we assume that $K$ is injective we can define $K^{-1}:\; \im(K) \to
L^2(\Omega)$, our problem now is to calculate $K^{-1}(g(t))$ where
$g(t)$ is our measure.

Recalling the SVE of $K$ we can write (for $g\in \im(K)$)
\begin{equation}
  \label{eq:inverseSVE}
  K^{-1}(g) = \sum _{j\in \mathbb{N}} v_j \frac{1}{\sigma _j} \ang{g,u_j}
\end{equation}
so $K^{-1}$ has a amplifying effect on the higher frequency components
of $g$.

We have (we won't prove this fact) that $g\in \im (K)$ if and only if
it satisfies the Picard condition:
\[ \sum _{j\in \mathbb{N}} \pa{ \frac{1}{\sigma _j} \ang{g,u_j}} ^2 <
\infty \] 

So if we measure $\tilde g + \varepsilon$ where $\varepsilon$ is a
small perturbation given by sperimental errors we cannot expect
$\tilde g$ to satisfy Picar condition, so simply inverting $K$ without
any regularization would be useless.


\subsubsection{Discretization}
\label{sec:fredholmdiscretization}

To to solve numerically the Fredholm problem we have do discretize it,
we can choose to regularize the problem before or after the
regularization.

Common discretization methods are collocation methods and projection
methods.

If we discretize the problem before regularizing it we'll get a bad
conditioned matrix, so we have reduced the original regularization
problem to the regularization of a bad conditioned matrix.









%\bibliographystyle{alpha}
%\bibliography{}


\end{document}

